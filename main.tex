%% start of file `template.tex'.
%% Copyright 2006-2015 Xavier Danaux (xdanaux@gmail.com), 2020-2021 moderncv maintainers (github.com/moderncv).
%
% This work may be distributed and/or modified under the
% conditions of the LaTeX Project Public License version 1.3c,
% available at http://www.latex-project.org/lppl/.


\documentclass[11pt,a4paper,sans]{moderncv}        % possible options include font size ('10pt', '11pt' and '12pt'), paper size ('a4paper', 'letterpaper', 'a5paper', 'legalpaper', 'executivepaper' and 'landscape') and font family ('sans' and 'roman')

% moderncv themes
\moderncvstyle{classic}                             % style options are 'casual' (default), 'classic', 'banking', 'oldstyle' and 'fancy'
\moderncvcolor{blue}                               % color options 'black', 'blue' (default), 'burgundy', 'green', 'grey', 'orange', 'purple' and 'red'
\usepackage[scale=0.65]{geometry}


% personal data
\name{Bruno}{GIARRIZZO}
%\title{Résumé title}                               % optional, remove / comment the line if not wanted
%\born{test}                                        % optional, remove / comment the line if not wanted
%\address{street and number}{postcode city}{country}% optional, remove / comment the line if not wanted; the "postcode city" and "country" arguments can be omitted or provided empty
%\phone[mobile]{+1~(234)~567~890}                   % optional, remove / comment the line if not wanted; the optional "type" of the phone can be "mobile" (default), "fixed" or "fax"
%\phone[fixed]{+2~(345)~678~901}
%\phone[fax]{+3~(456)~789~012}
%\email{john@doe.org}                               % optional, remove / comment the line if not wanted
%\homepage{www.johndoe.com}                         % optional, remove / comment the line if not wanted

%\extrainfo{additional information}                 % optional, remove / comment the line if not wanted
%\photo[64pt][0.4pt]{picture}                       % optional, remove / comment the line if not wanted; '64pt' is the height the picture must be resized to, 0.4pt is the thickness of the frame around it (put it to 0pt for no frame) and 'picture' is the name of the picture file

\quote{"Code is like humor. When you have to explain it, it's bad."}

\begin{document}

%-----       resume       ---------------------------------------------------------

\makecvtitle

Développeur maniant majoritairement le langage Python, je me forme au swift ainsi qu'aux technologies cloud, et ainsi envisager une carrière de « DevOps Fullstack ».\newline

Muni d'un lourd bagage d'administrateur système avec une formation en cybersécurité, j'ai pu travailler et/ou écrire du code pour des startups, des retailers, des institutions financières ainsi que des organismes territoriaux Français.

\section{Experience}

  \cventry{Sept 2020 -- Now}{Developer}{BNP Paribas - CIB}{Paris}{}{
    Au sein d'une équipe de 15 développeurs, travaillant en total remote, des intervenants répartis entre la France, l'Angleterre, l'Espagne, les USA et la Chine; J'ai la charge de deux applications :\newline{}
    \begin{itemize}
      \item La première permettant d'ordonnancer la mise à jour des différentes VM sur des timeslots choisis (Python 2.7, Django 1.11)
      \item La seconde est une API permettant de gérer l'inventaire des services Bare Metal (Python 3.7, Flask 1.0.2)
    \end{itemize}
    J'ai pu également participer à la refonte de l'application gestionnaire des produits logiciels gérée par la BNP (Python 3.9, FastAPI 0.63.0)
  }

  \cventry{Sept 2017 -- Nov 2019}{DevOps}{Motoblouz.com}{Carvin}{}{
    Au sein de l'équipe IT comptant 8 développeurs et 2 AdminSys/DevOps, AGILE : \newline{}
    \begin{itemize}
      \item Refonte de l'infrastructure technique.
      \item Automatisation complète des déploiements applicatifs.
      \item Automatisation de la configuration, des mises à jours des serveurs dans différents environnements (externe, prod, préprod)
      \item Mise en place d'une CI/CD complète (Lint/Tests/Build/Delivery), basée sur Gitlab-CI
    \end{itemize}
    Annexe : \newline{}
    \begin{itemize}
      \item Developpement de scripts d'imports de data pour la BI
      \item Conception d'une app permettant le déploiement de communications commerciales en magasin sur TV via raspberry PI + Ansible
      \item Conception d'un micro-service d'ouverture de tiroir caisse.
    \end{itemize}
  }

  \cventry{Apr 2017 -- Jul 2017}{SysOps}{Altima Agency}{Roubaix}{}{
    Exploitation au sein du service Hosting :\newline{}
    \begin{itemize}
      \item Gestions des tickets niveau 1 à 3 et incidents
      \item Création, modification, Suppression de compte LDAP
      \item Gestion des sauvegardes
      \item Provisionnement de serveurs
      \item Admin HyperViseur VMWare
    \end{itemize}
  }

  \cventry{Nov 2015 -- Apr 2017}{Chef de Projet junior}{Decathlon}{Lille}{}{
    Mes missions principales etaient :\newline{}
    \begin{itemize}
      \item Déploiement d'une passerelle d'accès sécurisé vers les serveurs hébergeant des données à caractère personnel.
      \item Déploiement d'une solution de chiffrement des données à caractère personnel.
    \end{itemize}
  }

  \cventry{Jan 2015 -- Aug 2015}{SysSecOps}{Motoblouz.com}{Carvin}{}{
    \begin{itemize}
      \item Déploiement d'authentificiation Wifi basée sur un annuaire Active Directory. 
      \item Audit de sécurité de l'infrastructure externe, puis de la pré-production, et enfin, sur des périmètres définis de la production. 
      \item Analyse forensic d'un des serveurs sur infra externe ayant été attaqué, Infecté, puis application des contre-mesures.
    \end{itemize}
  }

  \cventry{Sept 2013 -- Aug 2014}{SysAdmin}{Mairie de Denain}{Denain}{}{
    Assistance de l'administrateur en chef dans les tâches de gestion du parc.\newline{}
    \begin{itemize}
      \item Déploiement de solution de monitoring basée sur le couple Nagios \& Centreon afin de verifier la connectivité et la performance des liens des sites distants. 
      \item Mise en place d'une solutions de gestion de parc (OCS Inventory \& GLPI) puis d'une mise en place d'un PRA (Plan de Reprise d'activité) inexistant jusqu'alors. 
      \item Développer l'utilisation de clients légers dans une optique de réduction des coûts matériels.
    \end{itemize}
  }

\pagebreak

\section{Skill matrix}
\cvskilllegend*[1em]{}% adjust post spacing
\cvskillhead[-0.1em]%   this inserts the standard legend in english and adjust padding
\cvskillentry*{Language}{3}{Bash}{10}{Réalisation de scripts d'adminsys divers et variées}
\cvskillentry{}{4}{Python}{5}{Langage de prédilection, utilisé dans mes différentes experiences, et autres projets personnel}
\cvskillentry{}{1}{Swift}{0,5}{Autoformation, création d'API rest (vapor), création d'apps iOS/iPadOS/macOS (swiftUI et Catalyst)}
\cvskillentry*{VCS/CI}{3}{Github + Github Actions}{2}{Cadre Perso : Hosting de projet perso, execution pipeline CI de tests/build/push}
\cvskillentry{}{3}{Gitlab + GitlabCI}{5}{Cadre pro : Utilisation CI complète, Création \& refacto CI existante}
\cvskillentry*{DevOps}{3}{Docker}{4}{Création d'image, optimisation}
\cvskillentry{}{3}{Ansible}{4}{Création de playbook de configuration de serveurs et déploiement applicatifs}
\cvskillentry{}{2}{kubernetes}{1}{Déploiement d'applicatifs sur cluster, et administration basique}
\cvskillentry*{Cloud}{2}{Scaleway}{2}{Hosting de projets personnels (Instance classique + Kubernetes Kapsule + S3)}
\cvskillentry{}{2}{Amazon}{0,5}{Autoformation}
\cvskillentry{}{2}{Google}{0,5}{Autoformation}
\cvskillentry*{OS}{3}{Linux}{15}{Utilisateur de Linux depuis 15 ans, et l'utilisant dans le cadre pro depuis 12 ans}
\cvskillentry*[1em]{Methods}{4}{Agile}{5}{Travail en mode Agile depuis 2017}


\section{Languages}
\cvitemwithcomment{French}{Mother Language}{}
\cvitemwithcomment{English}{Fluent}{}
\cvitemwithcomment{Italian}{Basic Usage in family}{}


\end{document}
